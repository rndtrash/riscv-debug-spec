\chapter{Вступление}
\label{sec:intro}

Как только проект прогрессирует из симуляции в реализацию на ``железе'', у пользователя
существенно снижается уровень контроля и понимания текущего состояния системы.
Чтобы помочь в создании и отладке низкоуровневого ПО и АО,
важно иметь хорошую поддержку отладки, встроенную в само ядро.
Когда на ядре работает надёжная ОС, ПО может взять на себя множество
задач отладки. Однако в большинстве случаев важна поддержка в самом оборудовании.

Данный документ излагает стандартную архитектуру для поддержки отладки
на платформах с ядром RISC-V. Данная архитектура допускает множество реализаций и
компромиссов, что приветствуется в широком спектре реализаций RISC-V.
В то же время эта спецификация определяет общие интерфейсы, позволяющие
отладчикам и компонентам поддерживать множество аппаратных платформ
на основе архитектуры набора команд RISC-V.

Разработчики систем могут добавить дополнительную аппаратную поддержку отладки,
но эта спецификация определяет только стандартный интерфейс для базового
функционала.

\section{Терминология}

\begin{description}[style=nextline]
    \item[AMO]
        Atomic Memory Operation - атомарная операция над памятью.
    \item[BYPASS]
        Инструкция JTAG, предназначенная для выбора регистра обхода в качестве регистра данных, также называемая BYPASS.
    \item[компонент]
        Ядро RISC-V или любая другая часть аппаратной платфомы.
        Как правило, все компоненты будут подключены к одной системной
        шине.
    \item[CSR]
        Control and Status Register - регистр контроля/статуса.
    \item[DM]
        Debug Module - отладочный модуль (см. Раздел \ref{dm}).
    \item[DMI]
        Debug Module Interface - интерфейс отладочного модуля (см. Раздел \ref{dmi}).
    \item[DR]
        JTAG Data Register - регистр данных JTAG.
    \item[DTM]
        Debug Transport Module - модуль передачи отладочной информации (см. Раздел \ref{dtm}).
    \item[DXLEN]
        Debug XLEN - XLEN Отладчика, являющийся самым широким XLEN,
        поддерживаемым hart, игнорируя текущее значение \Fmxl в \Rmisa.
    \item[GPR]
        General Purpose Register - регистр общего назначения.
    \item[аппаратная платформа]
        Система, состоящая из одного или нескольких \emph{компонентов}.
    \item[hart]
        HARdware Thread - аппаратный поток в ядре RISC-V.
    \item[IDCODE]
        32-bit Identification CODE - 32-битный идентифицирующий код, а также инструкция JTAG,
        возвращающая значение IDCODE.
    \item[IR]
        JTAG Instruction Register - регистр инструкции JTAG.
    \item[JTAG]
        Имеется ввиду работа, проделанная IEEE Joint Test Action Group, описанная в
        IEEE 1149.1.
    \item[NAPOT]
        Naturally Aligned Powers-Of-Two - выровненный по степени двойки.
    \item[NMI]
        Non-Maskable Interrupt - немаскируемое прерывание.
    \item[физический адрес]
        Адрес, непосредственно используемый в системной шине.
    \item[SBA]
        System Bus Access - доступ к системной шине (см. Раздел \ref{systembusaccess}).
    \item[TAP]
        Test Access Port - порт тестирования. Определение дано в IEEE 1149.1.
    \item[TM]
        Trigger Module - модуль триггера (см. Раздел \ref{tm}).
    \item[виртуальный адрес]
        Адрес с точки зрения hart. Если hart использует преобразование адресов, то
        виртуальный адрес может отличаться от физического. Если не выполнены никакие преобразования,
        то адреса будут одинаковыми.
    \item[\Rxepc]
        CSR со счётчиком команд исключения (например \Rmepc), применимый в
        режиме, в который вводит отладочная ловушка.
\end{description}

\section{Контекст}

\begin{steps}{Документ использует термины, описанные в:}
\item The RISC-V Instruction Set Manual, Volume I: User-Level ISA, Document
    Version 2.2 (the ISA Spec)
\item The RISC-V Instruction Set Manual, Volume II: Privileged Architecture,
    Version 1.12 (the Privileged Spec)
\end{steps}

\subsection{Версии}

Версия 0.13 данного документа была утверждена комиссией фонда RISC-V.
Версии 0.13.$x$ являются исправлениями ошибок в той утверждённой спецификации.

Версия 0.14 была черновиком, который никогда не был официально ратифицирован.

Версия 1.0.0 почти что полностью прямо и обратно совместима с версией 0.13.

\newcommand{\PR}[1]{\href{https://github.com/riscv/riscv-debug-spec/pull/#1}{\##1}}

\subsubsection{Исправления ошибок с 0.13 по 1.0}

\begin{steps}{Changes that fix a bug in the spec:}
    \item Fix order of operations described in \RdmSbdataZero. \PR{392}
    \item Resume ack is set after resume, in Section~\ref{runcontrol}. \PR{400}
    \item \FcsrTextraThirtytwoSselect applies to \FcsrTextraThirtytwoSvalue. \PR{402}
    \item \FcsrTcontrolMte only applies when action=0. \PR{411}
    \item \FacAccessmemoryAamsize does not affect Argument Width. \PR{420}
    \item Clarify that harts halt out of reset if \FdmDmcontrolHaltreq=1. \PR{419}
\end{steps}

\subsubsection{Несовместимые изменения с 0.13 по 1.0}

\begin{steps}{Changes that are not backwards-compatible. Debuggers or
hardware implementations that implement 0.13 will have to change something in
order to implement 1.0:}
    \item Make haltsum0 optional if there is only one hart. \PR{505}
    \item System bus autoincrement only happens if an access actually
    takes place. (\RdmSbdataZero) \PR{507}
    \item Bump \FdmDmstatusVersion to 3. \PR{512}
    \item Require debugger to poll \FdmDmcontrolDmactive after lowering it. \PR{566}
    \item Add \FcsrIcountPending to \RcsrIcount. \PR{574}
\end{steps}

\subsubsection{Небольшие изменения с 0.13 по 1.0}

\begin{steps}{Changes that slightly modify defined behavior. Technically backwards
incompatible, but unlikely to be noticeable:}
    \item \FcsrDcsrStopcount only applies to hart-local counters. \PR{405}
    \item \FdmDmstatusVersion may be invalid when \FdmDmcontrolDmactive=0. \PR{414}
    \item Address triggers (\RcsrMcontrol) may fire on any accessed address. \PR{421}
    \item All trigger registers (Section~\ref{csrTrigger}) are optional. \PR{431}
    \item When extending IR, \RdtmBypass still is all ones. \PR{437}
    \item \FcsrDcsrEbreaks and \FcsrDcsrEbreaku are WARL. \PR{458}
    \item NMI are disabled by \FcsrDcsrStepie. \PR{465}
    \item R/W1C fields should be cleared by writing every bit high. \PR{472}
    \item Specify trigger priorities in Table~\ref{tab:priority} relative to exceptions. \PR{478}
    \item Time may pass before \FdmDmcontrolDmactive becomes high. \PR{500}
    \item Clear MPRV when resuming into lower privilege mode. \PR{503}
    \item Halt state may not be preserved across reset. \PR{504}
    \item Hardware should clear trigger action when \FcsrTdataOneDmode is
        cleared and action is 1. \PR{501}
    \item Change quick access exceptions to halt the target in
    Section~\ref{acQuickaccess}. \PR{585}
    \item Writing 0 to \RcsrTdataOne forces a state where \RcsrTdataTwo and
        \RcsrTdataThree are writable. \PR{598}
\end{steps}

\subsubsection{Новый функционал с 0.13 по 1.0}

\begin{steps}{New backwards-compatible feature that did not exist before:}
    \item Add halt groups and external triggers in Section~\ref{hrgroups}. \PR{404}
    \item Reserve some DMI space for non-standard use. See \RdmCustom, and
        \RdmCustomZero through \RdmCustomFifteen. \PR{406}
    \item Reserve trigger \FcsrTdataOneType values for non-standard use. \PR{417}
    \item Add \FcsrEtriggerNmi bit to \RcsrEtrigger. \PR{408}
    \item Recommend matching on every accessed address. \PR{449}
    \item Add resume groups in Section~\ref{hrgroups}. \PR{506}
    \item Add \FdmAbstractcsRelaxedpriv. \PR{536}
    \item Move \RcsrScontext, renaming original to \RcsrMscontext, and create
        \RcsrHcontext. \PR{535}
    \item Add \RcsrMcontrolSix, deprecating \RcsrMcontrol. \PR{538}
    \item Add hypervisor support: \FcsrDcsrEbreakvs, \FcsrDcsrEbreakvu,
        \FcsrDcsrV, \RcsrHcontext, \RcsrMcontrol, \RcsrMcontrolSix, and
        \RvirtPriv. \PR{549}
    \item Optionally make \FdmDmstatusAnyunavail and \FdmDmstatusAllunavail
    sticky, controlled by \FdmDmstatusStickyunavail. \PR{520}
    \item Add \RcsrTmexttrigger to support trigger module external trigger
        inputs. \PR{543}
    \item Describe \RcsrMcontrol and \RcsrMcontrolSix behavior with atomic instructions. \PR{561}
    \item Trigger hit bits must be set on fire, may be set on match. \PR{593}
    \item Add \FcsrTextraThirtytwoSbytemask and \FcsrTextraSixtyfourSbytemask
        to \RcsrTextraThirtytwo and \RcsrTextraSixtyfour. \PR{588}
    \item Allow debugger to request harts stay alive with keepalive bit in
        Section~\ref{keepalive}. \PR{592}
    \item Add \FdmDmstatusNdmresetpending to allow a debugger to determine
        when ndmreset is complete. \PR{594}
    \item Add \FcsrTmexttriggerIntctl to support triggers from an interrupt controller. \PR{599}
\end{steps}
\section{Об этом документе}

\subsection{Структура}

Этот документ состоит из двух частей. Главная часть документа - это
спецификация, данная в нескольких главах. Во второй части данного документа
находится приложение. Информация в приложении предназначена для уточнения
и предоставления примеров, но это не является частью самой спецификации.

\subsection{ISA против не-ISA}

Эта спецификация состоит как из частей, входящих в ISA, так и наоборот.
Части ISA определяют состояние и поведение hart, а части не-ISA - состояние и поведение
всего остального
Главы, в которых содержатся только части, (не)относящиеся к ISA, отмечены как таковые
в заголовке. Главы без уточнения в заголовке относятся как к ISA, так и к не-ISA.

\subsection{Формат определений регистров}

Все определения регистров в этом документе следуют показанному ниже формату.
Простой график показывает положение полей в регистре. Номер верхнего и нижнего бита
отмечен соответственно в верхнем левом и верхнем правом углу каждого поля. Суммарное количество
битов в поле написано под ним.

После графика следует таблица, обозначающая имя, описание, разрешённый доступ и
значение сброса для каждого поля. Разрешённые доступы перечислены в Таблице~\ref{tab:access}.
Значение сброса является либо константой, либо ``Предустановленно.''
Второе обозначение значит, что регистр имеет действительное значение, зависящее
от реализации.

Имена регистров и их поля являются гиперссылками на их определение. Также они
находятся в перечне на странице \pageref{index}.

\input{sample_registers.tex}

\begin{table}[htp]
    \centering
    \caption{Аббревиатуры доступа к регистру}
    \label{tab:access}
    \begin{tabulary}{\textwidth}{|l|L|}
        \hline
        R & Read-only - только для чтения. \\
        \hline
        R/W & Read/Write - чтение/запись. \\
        \hline
        R/W1C & Read/Write Ones to Clear - чтение/запись, единицы для очистки.
        Запись нуля в каждый бит не имеет какого либо эффекта.
        Запись единицы в каждый бит очищает поле. Результат остальных операций
        не определён. \\
        \hline
        WARZ & Write any, read zero - любая запись, чтение - ноль.
        Отладчик может записать любое значение. При чтении
        это поле возвращает 0. \\
        \hline
        W1 & Write-only - только для записи.
        Только запись единиц влияет на значение. При чтении возвращённое
        значение должно быть равным нулю. \\
        \hline
        WARL & Write any, read legal - любая запись, чтение допустимого [значения].
        Отладчик может записать любое значение. Если значение
        не поддерживается, реализация преобразовывает значение в поддерживаемое. \\
        \hline
    \end{tabulary}
\end{table}

%\section{Порядок прочтения}

В этом разделе описаны минимальные части спецификации, необходимые для понимания
данного куска функциональности. Естественно, лучше прочитать всё, но чтение одного раздела 
может больше помочь в начале работы, чем если бы вы читали всё от начала до конца.
Независимо от того, о чем вы хотите узнать, лучше всего прочитать Раздел~\ref{overview}
в первую очередь.

\subsubsection{Halt/Resume}

Sections \ref{dmi}, \ref{selectingharts}, \ref{haltcontrol}, \ref{dmcontrol}, \ref{dmstatus},
\ref{haltsum}, \ref{deb:halt}.

\subsubsection{Abstract Register Access}

Sections \ref{abstractcommands}, \ref{abstractcs}, \ref{command},
\ref{data0}, \ref{deb:abstractreg}.

\subsubsection{Program Buffer}

Sections \ref{programbuffer}, \ref{progbufcs}, \ref{progbuf0}, \ref{access
register}, \ref{debugmode}, \ref{deb:regprogbuf}, \ref{deb:mrprogbuf}.

\subsubsection{JTAG Debug Transport Module}

Sections \ref{dtm}, \ref{jtagdtm}, \ref{dbusaccess}.

\subsubsection{System Bus Master}

Sections \ref{systembusaccess}, \ref{sbcs}, \ref{sbaddress0}, \ref{sbaddress1},
\ref{sbaddress2}, \ref{sbdata0}, \ref{sbdata1}, \ref{sbdata2}, \ref{sbdata3},
\ref{deb:mrsysbus}.

\subsubsection{Reset}

TODO

\subsubsection{Security}

TODO

\subsubsection{Triggers}

TODO

\subsubsection{Serial Ports}

TODO



\section{Предпосылки}

Существует множество применений для отдельного аппаратного отладчика, как
встроенного в ядро процессора, так и подключаемого извне.
Эта спецификация адресована сценариям использования, перечисленным ниже. Имплементации
могут не поддерживать каждую возможность, так что некоторые варианты использования
могут не поддерживаться.

\begin{itemize}

\item Отладка низкоуровневого ПО при отсутствии ОС или других програм.

\item Отладка проблем в самой ОС.

\item Инициализация аппаратной платформы для тестирования, конфигурации, а также программных
  компонентов до того, как код будет исполнен на аппаратной платформе.

\item Получение доступа к компонентам аппаратной платформы без работающего ЦП.

\end{itemize}

Также даже без аппаратного интерфейса отладки архитектурная
поддержка в процессоре RISC-V поможет в отладке приложений и анализе
производительности с помощью аппаратных триггеров и точек останова.

\section{Возможности}

Отладочный интерфейс, описанный в данной спецификации, поддерживает следующие функции:

\begin{enumerate}
   \item Все регистры hart (включая CSR) могут быть прочитаны/перезаписаны.
   \item Имеется доступ к памяти как с точки зрения hart, так и напрямую через
       системную шину.
   \item Существует поддержка RV32, RV64 и будущего RV128.
   \item Любой hart в аппаратной платформе может быть отдельно отлажен.
   \item Отладчик способен найти почти что\footnote{Примечательным исключением
       является информация о схеме распределия памяти и переферии.} всё, что ему
       нужно знать сам по себе, без конфигурации пользователем.
   \item Каждый hart может быть отлажен начиная с самой первой выполненной инструкции.
   \item Hart в RISC-V можно приостановить, когда исполнена инструкция программной
       точки останова.
   \item Железо может исполнить инструкции пошагово.
   \item Функционал отладчика независим от используемого интерфейса передачи отладочных данных.
   \item Отладчику не нужно что-либо знать о микроархитектуре отлаживаемого hart.

   \item Работа произвольного набора hart может быть одновременно приостановлена или
       возобновлена. (Опционально)
   \item Можно выполнить произвольные инструкции на
Arbitrary instructions can be executed on
       приостановленном hart. Это значит, что не требуется никакого нового функционала
       отладчика, когда ядро имеет дополнительные или особые инструкциии или состояния,
       если существует программы, способные перемещать это состояние в GPR. (Опционально)
   \item Возможно получить доступ к регистрам без приостановки исполнения кода. (Опционально)
   \item Работающий hart может быть адресован для исполнения короткого набора
       инструкций с небольшими издержками. (Опционально)
   \item Управляющий системной шиной ({\tt нужна помощь с переводом ``system bus master''}) допускает
       доступ к памяти без использования каких либо hart. (Опционально)
   \item Hart в RISC-V может быть приостановлен, когда выполняется условие триггера на
       счётчик инструкций, запись/чтение адреса/данных, или особый опкод инструкции. (Опционально)
    \item Hart-ы могут быть сгруппированы, и все hart-ы, принадлежащие одной группе приостановятся,
        когда хотя бы один из них был приостановлен. Эти группы могут вызывать или реагировать на
        внешние триггеры. (Опционально)
\end{enumerate}

Этот документ не предлагает стратегию или реализацию техник для тестирования, отладки или
отлавливания ошибок в АО. Сканирование, встроенная самодиагностика (built-in self test, BIST) и прочее
не рассматриваются в этой спецификации, но она не ограничивает их использование в системах на основе
RISC-V.

Существует возможность отладки кода, использующего программные потоки, но для них нет
особой поддержки отладчиком.
